\subsection*{Què pot fet una app bancari?}
\addcontentsline{toc}{subsection}{Què pot fet una app bancari?}

Una "app bancària" es refereix a una aplicació mòbil dissenyada i desenvolupada per una entitat bancària perquè els seus clients puguin accedir i gestionar els seus serveis financers a través de dispositius mòbils, com ara telèfons intel·ligents o tauletes. Aquestes aplicacions ofereixen una interfície fàcil d'utilitzar que permet als usuaris realitzar una varietat d'operacions bancàries des de la comoditat dels seus dispositius mòbils. Es implementen mesures de seguretat com autenticació de dos factors, biometria (empremtes digitals, reconeixement facial) i xifrat de dades per protegir la informació de l'usuari.

Les característiques comunes de les apps bancàries inclouen:

\begin{itemize}
    \item Consulta de Saldo i Moviments: Veure el saldo dels comptes i revisar les transaccions recents.
    \item Transferències: Fer transferències entre comptes propis o a comptes de tercers.
    \item  Pagaments: Pagar factures i fer altres pagaments.
    \item Alertes i Notificacions: Configurar alertes per a transaccions específiques o rebre notificacions sobre l'estat del compte.
    \item Inversions: Accedir a informació sobre inversions i gestionar carteres.
    \item Targetes: Gestionar targetes de dèbit o crèdit, bloquejar-les en cas de pèrdua, etc.
    \item Servei al Client: En alguns casos, accedir a serveis d'atenció al client a través de l'aplicació.
\end{itemize}



\subsection*{OpenBanK}
\addcontentsline{toc}{subsection}{OpenBanK}





\subsection*{CaixaBanK}
\addcontentsline{toc}{subsection}{CaixaBanK}




+ i un banc estranger 



3. seguretat de banca online
    - proteccion de dados
    - evitar fraude
    - cripografia
