



\subsection*{Seguretat de dades del bancari}
\addcontentsline{toc}{subsection}{Seguretat de dades del bancari}


En general, els serveis financers solen implementar mesures de seguretat estàndard per protegir la informació financera i les transaccions. Aquestes mesures poden incloure:

\begin{itemize}
    \item Xifratge de dades en trànsit: S'utilitza xifratge de capa de transport (TLS o SSL) per protegir la informació mentre es transmet entre el dispositiu de l'usuari i els servidors del servei.
    \item Xifratge de Dades en Repòs: La informació emmagatzemada als servidors es xifra per protegir-la contra l'accés no autoritzat.
    \item Autenticació de Dos Factors: Per accedir al compte o fer transaccions, es pot requerir l'autenticació de dos factors, com ara la combinació d'una contrasenya amb un codi enviat al telèfon mòbil de l'usuari.
\end{itemize}

\subsection*{Seguretat de dades del client}
\addcontentsline{toc}{subsection}{Seguretat de dades del client}


La protecció de seguretat de dades és de feina dels entitats bancaris i també són dels usuaris, és imprescindible proporcionar informació i recursos educatius als usuaris per ajudar-los a comprendre la importància de pràctiques segures. Aquí és on entren en joc altres factors com:
\begin{itemize}
    \item No comparteixis les contrasenyes. Com més persones la sàpiguen, més vulnerable seràs a atacs, ja que aquestes persones podrien no guardar-les de manera segura, facilitant així que el compte sigui objecte d'un atac.
    \item Canvia les contrasenyes cada cert temps o revisa si els serveis i les pàgines en què et vas donar d'alta han estat objecte d'alguna filtració.
    Utilitza una contrasenya diferent per a cada web o, si més no, tingues una varietat de contrasenyes diferents. Així, en cas que descobreixin la teva clau, només podran accedir a unes poques webs.
    \item Mai utilitzis la contrasenya del teu correu electrònic com a clau segura d'altres webs. El teu correu sol ser la manera de resetejar les claus de qualsevol servei i estaràs fent que sigui més fàcil hackejar-lo. El ciberdelinqüent que accedeixi a una contrasenya de qualsevol servei provarà automàticament a introduir-la al correu electrònic que apareix subscrit.
    \item Activa l'autenticació en dos factors sempre que sigui possible. Això afegeix una capa de protecció addicional si algú descobreix la contrasenya.
\end{itemize}